\renewcommand{\baselinestretch}{1.5}
\chapter{Abstract}
\renewcommand{\baselinestretch}{\mystretch}

%\setlength{\parindent}{2em}
Fuzz testing uses the random generalised data (inputs) to detect the bugs or code error in the software \cite{miller2007fuzz}.
% JW: It's not important to describe the history of fuzzing in the abstract. Only the most important parts should go into your abstract.
% JW: Citations go before the full-stop. So you should write, for example, "in 1988 \cite{miller2007fuzz}."
This thesis summarises 
% JW: Use present or future tense in the abstract, not past tense. So it should be "summarises".
the research project of fuzzing test in hardware synthesis, especially in netlists. Typically, the main application of fuzzing algorithm is used in software testing. Applying it to hardware synthesis software testing is a effective way to obtain the potential crashes or bugs in these synthesis tools. There are two main contribution of this project, the fuzz testing algorithm
% JW: algorithm
has been advanced and the tested. This project achieve the....
In addition, 
% JW: "the tested" ??
% JW: The abstract needs to be self-contained, so you can't end with "the following parts".

%The abstract is written here. This should concisely state the main aim of the work and the achievements.







