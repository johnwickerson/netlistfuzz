\chapter{Project Design}
\renewcommand{\baselinestretch}{\mystretch}
\label{chap:ProDes}
\PARstart{T}{his} chapter illustrate the design process of this project. The first section 3.1 is divided into 2 subsection 3.1.1 and 3.1.2 to explains the core objectives and extensive objective this project. Next, methodology section 3.2 explains the algorithm design in flow chart and how to construct the developing environment. The final section discuss how to evaluate the result and performance of the designed fuzzer.
\section{Objectives}
\subsection{Core}
The core objective of this project is to write a program to generate simple combinational logic circuits (no clock signal) without sequential elements (flip-flop, register, ROM). A LINUX shell program need to be programmed to input the fuzzing netlist file to ABC.
\subsection{Extension}
The extension objective is constructing more complex circuit (not guaranteed to be supported by ABC). The generated circuit will be re-synthesized using the circuit folding method. Using sequential (Synchronous, Asynchronous, more than one clock signal) flip-flop, register, adder, mux and other logical elements. The input/output can be realized in tri-state. It indicates that the output port was allowed to yield high impedance state (z) and unknown state (x), in addition to the 1 and 0 logic states. 
\section{Methodology}
To achieve the aims and objective of this project, the research would be initially conducted on theoretical analysis. Understanding the basic principle of random variable generation and rules of Verilog file. To be specific, the first step is to analyze the synthesis principle of netlist file. For example, read the design specification from file and manual. Secondly, the tested hardware synthesis tool needs to be modified to accepted thousands of random inputs and output the results. The test target ABC is an open source software and its code can be modified by using C language compiler. The linux system is an effective environment to compile an open source software. As a consequence, the development of this fuzzing test will be implemented on Linux. Refer to the existing fuzzer, the fuzzer may also be compiled in C language by GNC compiler collection (GCC). For the extended objectives, the circuit optimization method is also required.

The practical implementation will be conducted on the
\section{Evaluation strategy}
The result will be evaluated in two aspect: qualities and quantities.

\begin{equation}
       \sum_{i=1}^{x} 
       \begin{pmatrix}
       n\\
       i\\
       \end{pmatrix}
       \theta^{i}(1-\theta)^{n-i}\ge\alpha
\end{equation}

